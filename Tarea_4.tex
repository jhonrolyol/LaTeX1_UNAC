% Preámbulo
\documentclass[12pt]{article}
% Paquetes
\usepackage{amsmath}
\usepackage[utf8]{inputenc}
\usepackage{blindtext}
\usepackage[spanish]{babel}
\usepackage{authblk}
\usepackage[T1]{fontenc}
\usepackage{biblatex}
\usepackage{refenums}
\usepackage{refstyle}
\usepackage{refcount} % Revisar la documentación
\usepackage{bibleref}
\usepackage{biblatex-bookinarticle}
% Información del documento
\title{Titulo del Articulo} % Título
\author{Autor} % Autor
\date{\today} % Fecha
% Entorno del documento
\begin{document}

     \maketitle
     
     \begin{abstract}
      Resumen del articulo.
     \end{abstract}
     
     \section{Una primera sección}
     Esta es la primera sección del articulo.
     
     \subsection{Subsección}
     Una sección dentro de una sección se denomina subsección.
     
     \subsubsection{Subsubsección}
     Esto es una sección dentro de una subsección, o sea una subsubsección.
     
     \paragraph{Párrafo}
     Esto corresponde a un párrafo resaltado. En los artículos ~\ref{1, 2, 3} se explican los procedimientos a seguir.

     \begin{thebibliography}{00}
         \bibitem{1} Referencia numero uno.
         \bibitem{2} Referencia numero dos.
         \bibitem{3} Referencia numero tres.
     \end{thebibliography}

\end{document}

