\documentclass{article}
\usepackage{authblk}
\usepackage{amsmath}
\usepackage{xcolor}
\usepackage{graphicx}
\title{An important paper}

\author[$1$]{R. Campbell}
\author[$2$]{M. Dane}
\author[$3$]{J. Jones}

\affil[$1$]{Department of Mathematics, Pennsylvania State University,Pittsburgh, Pennsylvania 13593}
\affil[$2$]{Atmospheric Research Station,
Pala Lundi, Fiji}
\affil[$3$]{Department of Philosophy, Freedman College,
Periwinkle, Colorado 84320}

\begin{document}
\maketitle

\begin{abstract}
Los sistemas oscilantes idealizados que hasta ahora hemos visto no tienen fricción; no hay fuerzas no conservativas, la energía mecánica total es constante, y un sistema puesto en movimiento sigue oscilando eternamente sin disminución de la amplitud. \ldots
\end{abstract}

\section{Introducci\'on}
La disminución de la amplitud causada por fuerzas disipativas se denomina amortiguamiento, y el movimiento correspondiente se llama oscilación amortiguada~\cite{sears}.

\paragraph{Atenci\'on:}
Sin embargo, los sistemas del mundo real siempre tienen fuerzas disipativas, y las oscilaciones cesan con el tiempo, a menos que un mecanismo reponga la energía mecánica disipada
El resto de este artículo está organizado de la siguiente manera.
La section~\ref{trabajos} da cuenta de trabajos anteriores.
Nuestros nuevos y emocionantes resultados se describen en la 
sección ~\ref{results}.
Finalmente, en la seccion~\ref{conclusions} se dan a conocer las conclusiones.

\section{Trabajos previos}\label{trabajos}
Un ejemplo de \LaTeXe{} fue escrito por Gil en~\cite{Gil:02}.
Una f\'ormula muy importante que tenemos que recordar es
\begin{equation}\label{newto1}
m \ddot{\vec{r}}=\vec{F}_{res},
\end{equation}
donde $\vec{F}_{res}$ es la fuerza resultante. Otra manera de escribir la segunda ley de Newton es
\begin{equation}\label{newton2}
\frac{d\vec{p}}{dt}=\vec{F}_{res},
\end{equation}
donde $\vec{p}\equiv m\vec{v}$ es el momento lineal de la part\'{i}cula. La ecuacion (~\ref{newton2}) es la versio menos conocida en los libros universitarios para las especialidades de ingenieria.//
Recordemos que los vecores unitarios al multiplicarse vectorialmente $\hat{i} \times \hat{j} =\hat{k} $.
Otra de las operaciones m\'as comunes es $\sqrt[3]{x}$. La temperatura de hoy d\'{i}a en Lima es de 18 $^{\circ}$C.


$2\sqrt{2},\quad 2^2\sqrt{2-\sqrt{2}},\quad 2^3\sqrt{2-\sqrt{2+\sqrt{2}}},
\quad 2^4\sqrt{2-\sqrt{2+\sqrt{2+\sqrt{2+\sqrt{2}}}}},\ldots
$
converge a $\pi$. Ahora veamos algunas letras en griego como
$\psi(x)$ y $\Psi(x)$.

\begin{equation}
\overline{x+y+z+w}
\end{equation}
\begin{equation}
\overline{x+\underline{y+z}+w}
\end{equation}
\begin{equation}
\overbrace{x+\underbrace{y+z}+w}
\end{equation}
Euler demostró que la serie
$\sum_{n=1}^\infty\frac{1}{n^2}$
converge, pero además que:
\begin{equation}
\sum_{n=1}^\infty\frac{1}{n^2}=\frac{\pi^2}{6}
\end{equation}

Euler demostró que la serie $\sum\limits_{n=1}^\infty\frac{1}{n^2}$
converge, pero además que:
\begin{equation}
\sum\nolimits_{n=1}^\infty\frac{1}{n^2}=\frac{\pi^2}{6}
\end{equation}

Así, $\lim\limits_{x\to\infty}\int_0^x\frac{\sin x}{x}\,\mathrm{d}x
=\frac{\pi}{2}$ y por definición,
\begin{equation}
\int_0^\infty\frac{\sin x}{x}\,\mathrm{d}x=\frac{\pi}{2}
\end{equation}

Así, $\lim\limits_{x\to\infty}\int_0^x\frac{\sin x}{x}dx
=\frac{\pi}{2}$ y por definición,
\begin{equation}
\int_0^\infty\frac{\sin x}{x}\,dx=\frac{\pi}{2}
\end{equation}

En caso de que se quiera colocar varias líneas de subíndices, se puede utilizar el comando substack como en el siguiente ejemplo:
\begin{equation}
p_k(x,t)=\prod_{i=1}^{n}
\left(\frac{x-t_i}{t_k-t_i}\right)
\end{equation}

\begin{equation}
p_k(x)=\prod_{\substack{i=1\\i\ne k}}^n
\left(\frac{x-t_i}{t_k-t_i}\right)
\end{equation}

\begin{equation}
\iint f(x,y) dxdy \qquad \iiint f(x,y,z) dxdydz \qquad
\idotsint_M dx_1\dots dx_n 
\end{equation}

\begin{align}
\int_0^\infty \textcolor{blue}{f(x)d(x)} = \textcolor{red}{g(x)} + C
\end{align}

\begin{equation}
J=
\begin{pmatrix}
A_1 & A_2 & A_3 & A_4 \\
B_1 & B_2 & B_3 & B_4 \\
C_1 & C_2 & C_3 & C_4 \\
D_1 & D_2 & D_3 & D_4 \\
\end{pmatrix}=0
\end{equation}

%Terminemos con un gr\'afico de Albert Einstein:
%\begin{figure}
%\begin{center} 
%\bigskip
%\includegraphics[]{einsten1.jpg}
%\end{center}
%\caption{Albert Einstein y su famosa f\'ormula TeX/LaTeX.}
%\end{figure}



\section{Resultados}\label{results}
En esta sección describimos los resultados.

\section{Conclusiones}\label{conclusions}
Trabajamos duro y logramos muy poco.

\begin{thebibliography}{1}

\bibitem{sears}
Francis Sears, Mark Zemansky, Hugh D. Young, Roger A. Freedman-Fisica Universitaria con Fisica Moderna. Volumen 1-Pearson (2013).
\bibitem{Gil:02}
J.~Y. Gil.
\newblock {\LaTeXe} for graduate students.
\newblock manuscript, Haifa, Israel, 2002.
\end{thebibliography}
\end{document}
This is never printed


