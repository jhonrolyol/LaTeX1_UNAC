% Preámbulo
\documentclass[a4paper,11pt]{article}
% Paquetes
\usepackage[utf8]{inputenc}
\usepackage{amsmath,amssymb}
%\usepackage{amsfonts,latexsym}
\usepackage{amsthm}
\usepackage[spanish]{babel}
\usepackage{titlesec}

% Opening
\title{Capítulo 16: Ejemplos}
\author{Curso de \LaTeX}
\date{}

\addtolength{\textwidth}{2cm}
\addtolength{\hoffset}{-1cm}
\addtolength{\textheight}{2cm}
\addtolength{\voffset}{-1cm}

\renewcommand{\thesection}{\Roman{section}}

\renewcommand{\theequation}{\thesection-\arabic{equation}}

\titleformat{\section}[hang]
{\bfseries}{
\fbox{SECCIÓN \thesection} \hspace{3mm}}{0pt}
{\bfseries}

\begin{document}

\maketitle



\section{Algunas ecuaciones}

\textbf{1.-} \ Pista: combinar \texttt{split} y \texttt{equation} 
(para tener el número de ecuación centrado)
 
\begin{equation}
\begin{split}
H_c=& \frac{n_1!\,n_2!\,n_3!}
{n_1+n_2+n_3}\sum_i\left[\binom{n_1}{i}
\binom{n_2}{n_3-n_1+i}\right.\\[1mm]
&+\left.\binom{n_1-1}{i} \binom{n_2-1}{n_3-n_1+i}\right]
\end{split}\label{combinatoria}
\end{equation}

\medskip
Referencia cruzada a la ecuación \eqref{combinatoria}

\bigskip

\textbf{2.-} \ Construir con el entorno \texttt{align}:

\begin{align}
\gamma_x(t)&=(\cos tu+\sen tx,v), \label{x} \\
\gamma_y(t)&=(u,\cos tv+\sen ty), \label{y} 
\end{align}

\medskip
Referencia cruzada a la ecuación \eqref{x}; \ 
Referencia cruzada a la ecuación \eqref{y}

\bigskip

\textbf{3.-}  Entornos array anidados para matrix por bloques
\begin{equation}\label{matriz}
\left(
\begin{array}{c@{}c@{}c}
 \begin{array}{|cc|}\hline
  a_{11} & a_{12} \\
  a_{21} & a_{22} \\\hline
 \end{array} & \mathbf{0} & \mathbf{0} \\
 \mathbf{0} &
 \begin{array}{|ccc|}\hline
  b_{11} & b_{12} & b_{13}\\
  b_{21} & b_{22} & b_{23}\\
  b_{31} & b_{32} & b_{33}\\\hline
 \end{array} & \mathbf{0} \\
 \mathbf{0} & \mathbf{0} &
 \begin{array}{|cc|}\hline
  c_{11} & c_{12} \\
  c_{21} & c_{22} \\\hline
 \end{array} \\
\end{array}
\right)
\end{equation}

  Referencia cruzada a la ecuación \eqref{matriz}

\section{Más ecuaciones}

\textbf{4.-} \ Construir con el entorno \texttt{alignat}:
\begin{alignat}{3}
V_i & =v_i - q_i v_j, & \qquad X_i & = x_i - q_i x_j,
 & \qquad U_i & = u_i,
 \qquad \text{para $i\ne j$;} \notag \\
V_j & = v_j, & \qquad X_j & = x_j,
  & \qquad U_j & = u_j + \sum_{i\ne j} q_i u_i.
\end{alignat}


\bigskip

\textbf{5.-} \ Construir con el entorno \texttt{cases}:

\begin{equation}
f(x)=\begin{cases}
0 & \text{if} \quad 0\leq x\leq 1/2 \\
1 & \text{if} \quad 1/2\leq x\leq 1 \\
g(x) & \parbox[t]{5.5cm}{if $x>1$, we evaluate
the change of the step function with 
respect to the integration constant $c(x)$. 
This is done for every $\xi\in\aleph$}
\end{cases}\label{casos}
\end{equation}

\medskip
Referencia cruzada a la ecuación \eqref{casos};

\bigskip

\textbf{6.-} \ Etiquetas especiales con \verb+\tag{Etiqueta}+
\begin{align}
f(x) & =a\tag{linear}\\
g(x) & =dx^{2}+cx+b\tag{quadratic}\\
h(x) & =\sin x\tag{trigonometric}
\end{align}


\newtheorem{teo}{Teorema}[section]
\newtheorem{lem}{Lema}


\section{Manejo de teoremas}

\begin{teo}\label{detmat} 
Sea $B=(b_{ij})$ una matrix $n\times n$. Entonces:
\begin{equation}
\prod_{i\in n}
\left(\sum_{j\in n}b_{ij}\right) = 0
\end{equation}
\end{teo}

Aquí metemos una referencia cruzada al teorema \ref{detmat}

\begin{lem}\label{lematonto}
Ejemplo de Lema...
\end{lem}

\section{Más teoremas}

\begin{teo}\label{detvec} 
Sea $B=(b_{i})$ un vector en $\mathbb{R}^n$. Entonces:
\begin{equation}
\sum_{j\in n}b_{j} = 0
\end{equation}
\end{teo}

\begin{proof}
Haciendo uso del lema \ref{lematonto} y de la igualdad 
\eqref{combinatoria} la demostración es directa.
\end{proof}

\begin{lem}[Juanito y Pepito]
\label{lemanuevo}
Ejemplo de Lema con comentario adicional...
\end{lem}


\end{document}